\documentclass[../main.tex]{subfiles}
\chapter{Lösungsmethoden}
\label{c:losungsmethoden}

\section{Verfahren 1}

\begin{itemize}
	\item Funktion $B = f(B_0)$ aufstellen
	\begin{itemize}
		\item als anonyme Funktion, Funktionshandle
		\item hardgecoded, eingefroren
	\end{itemize}
	\item Startwert für $B_0$ annehmen
	\begin{itemize}
		\item bei erster Ausführung ist das ein geratener,geschätzter Wert
		\item bei jeder weiteren Ausführung kann man den Lösungswert des vorhergehenden Schrittes verwenden
	\end{itemize}
	\item Lösen der Funktion bis die Differenz $B - B_0$ ein Abbruchkriterium erreicht
\end{itemize}
