\documentclass[../main.tex]{subfiles}
\chapter{Lösungsmethoden}
\label{c:losungsmethoden}

\section[Verfahren 1]{Verfahren 1 - Iteration mit Polarkoordinaten}

\begin{itemize}
	\item Funktion $B = f(B_0)$ aufstellen
	\begin{itemize}
		\item als anonyme Funktion, Funktionshandle
		\item hardgecoded, eingefroren
		\item Verwendung der Formel mit Polarkoordinaten
	\end{itemize}
	\item Startwert für $B_0$ annehmen
	\begin{itemize}
		\item bei erster Ausführung ist das ein geratener,geschätzter Wert
		\item bei jeder weiteren Ausführung kann man den Lösungswert des vorhergehenden Schrittes verwenden
	\end{itemize}
	\item Lösen der Funktion bis die Differenz $B - B_0$ ein Abbruchkriterium erreicht
\end{itemize}

\section[Verfahren 2]{Verfahren 2 - Lösung mit \code{fzero}}

\begin{itemize}
	\item Funktion $f(B) = p_{Wst,z}(B) - z_{soll}$ aufstellen
	\begin{itemize}
		\item alle bekannten Werte substituieren mit \code{subs}
		\item Funktions-Handle aus symbolischer Funktion mit \code{matlabFunction}
	\end{itemize}
	\item Startwert für \code{fzero} annehmen
	\begin{itemize}
		\item bei erster Ausführung ist das ein geratener, geschätzter Wert
		\item bei jeder weiteren Ausführung kann man den Lösungswert des vorhergehenden Schrittes verwenden
	\end{itemize}
\end{itemize}

\section[Verfahren 3]{Verfahren 3 - Sinus Approximation}

\begin{itemize}
	\item Verwendung eines angenäherten Sinus
\end{itemize}

\section[Verfahren 4]{Verfahren 4 - Lösung mit \code{solve}}

\begin{itemize}
	\item Funktion $z_{soll} = p_{Wst,z}$ aufstellen
	\begin{itemize}
		\item bekannte Werte substituieren mit \code{subs}
	\end{itemize}
	\item lösen nach $B$ mit \code{solve(@(B) fun, B)}
\end{itemize}

\section[Verfahren 5]{Verfahren 5 - Newton Algorithmus}